%!TEX root = ts.tex

\rSec0[asymmetric]{Asymmetric Fence}

\rSec1[asymmetric.general]{General}

This clause describes Asymmetric Fence access. 


\rSec1[asymmetric.syn]{Header <experimental/asymmetric_fence> synopsis}


\begin{codeblock}
namespace std::experimental::inline concurrency_v2 {

// 7.3 asymmetric_thread_fence_heavy
void asymmetric_thread_fence_heavy(memory_order order) noexcept;

// 7.4 asymmetric_thread_fence_light
void asymmetric_thread_fence_light(memory_order order) noexcept;
}
\end{codeblock}

%%\uline{\texttt{\ namespace\ std::experimental::inline\ %%concurrency\_v2\ \{\ \ ~~void*\ %%atomic\_load\_per\_byte\_memcpy(void*\ dest,\ const\ void*\ %%source,\ size\_t\ count,\ memory\_order\ order);\ \ ~~void*\ %%atomic\_store\_per\_byte\_memcpy(void*\ dest,\ const\ void*\ %%source,\ size\_t\ count,\ memory\_order\ order);\ \ ~~\#define\ %%\_\_cpp\_lib\_experimental\_bytewise\_atomic\_memcpy\ \ %%202XYYL\ \}}}
%%\rSec1[bytewiseatomicmemcpy.general]{General}
\pnum

This subclause introduces synchronization primitives called \defn{heavyweight-fences} and
\defn{lightweight-fences}. Like fences, heavyweight-fences and lightweight-fences can have acquire
semantics, release semantics, or both, and can be sequentially consistent (in which case they
are included in the total order \tcode{S} on \tcode{memory_order::seq_cst} operations).
A heavyweight-fence has all the
synchronization effects of a fence (\CppXref{31.11} [atomics.fences]). \begin{note}  Heavyweight-fences and
lightweight-fences are distinct from fences. \end{note}

\pnum
A heavyweight-fence with acquire semantics is called an \defn{acquire heavyweight-fence}. A
heavyweight-fence with release semantics is called a \defn{release heavyweight-fence}. A
lightweight-fence with acquire semantics is called an \defn{acquire lightweight-fence}. A
lightweight-fence with release semantics is called a \defn{release lightweight-fence}.

\pnum
If there are evaluations A and B, and atomic operations X and Y, both operating on some atomic
object M, such that A is sequenced before X, X modifies M, Y is sequenced before B, and Y
reads the value written by X or a value written by any side effect in the hypothetical release
sequence X would head if it were a release operation, and one of the following hold:
\begin{itemize}
    \item  A is a release lightweight-fence and B is an acquire heavyweight-fence; or
\item  A is a release heavyweight-fence and B is an acquire lightweight-fence
\end{itemize}

then any evaluation sequenced before A strongly happens before any evaluation that B is
sequenced before.


\rSec1[asymmetric.heavy]{asymmetric_thread_fence_heavy}
\begin{itemdecl}
void asymmetric_thread_fence_heavy(memory_order order) noexcept;
\end{itemdecl}

\begin{itemdescr}


\pnum
\effects
Depending on the value of \tcode{order}, this operation:
\begin{itemize}
    \item  has no effects, if \tcode{order == memory\_order::relaxed};
\item  is an acquire heavyweight-fence, if \tcode{order == memory\_order::acquire} or \tcode{order ==
memory\_order::consume};
\item is a release heavyweight-fence, if \tcode{order == memory\_order::release};
\item is both an acquire heavyweight-fence and a release heavyweight-fence, if \tcode{order ==
memory_order::acq_rel};
\item is a sequentially consistent acquire and release heavyweight-fence, if \tcode{order ==
memory_order::seq_cst}.
\end{itemize}

\end{itemdescr}

\rSec1[asymmetric.light]{asymmetric_thread_fence_light}
\begin{itemdecl}
void asymmetric_thread_fence_light(memory_order order) noexcept;
\end{itemdecl}

\begin{itemdescr}


\pnum
\effects
Depending on the value of \tcode{order}, this operation:
\begin{itemize}
    \item  has no effects, if \tcode{order == memory\_order::relaxed};
\item  is an acquire lightweight-fence, if \tcode{order == memory\_order::acquire} or \tcode{order ==
memory\_order::consume};
\item is a release lightweight-fence, if \tcode{order == memory\_order::release};
\item is both an acquire lightweight-fence and a release lightweight-fence, if \tcode{order ==
memory\_order::acq\_rel};
\item is a sequentially consistent acquire and release lightweight-fence, if \tcode{order ==
memory\_order::seq\_cst}.
\end{itemize}

\end{itemdescr}



\begin{note} Delegating both heavyweight-fence and lightweight-fence functions to an
\tcode{atomic_thread_fence(order)} call is a valid implementation. Implementations can adopt
techniques in which calls to \tcode{asymmetric_thread_fence_light} execute more quickly than calls
to \tcode{atomic_thread_fence} with the same \tcode{memory_order}, at the cost of
\tcode{asymmetric_thread_fence_heavy} executing more slowly than calls to
\tcode{atomic_thread_fence} with the same \tcode{memory_order}
\end{note}  


