%!TEX root = ts.tex

\rSec0[atomiccpy]{Bytewise Atomic Memcpy}

\rSec1[atomiccpy.general]{General}

This clause describes bytewise atomic memcpy access. 


\rSec1[atomiccpy.syn]{Header <experimental/bytewise_atomic_memcpy> synopsis}
%%\uline{\textbf{Header
%%\texttt{\textless{}experimental/bytewise\_atomic\_memcpy\textgre%%ater{}}
%%synopsis}}

\begin{codeblock}
namespace std::experimental::inline concurrency_v2 {
  void* atomic_load_per_byte_memcpy(void* dest, const void* source, size_t count, memory_order order);
  void* atomic_store_per_byte_memcpy(void* dest, const void* source, size_t count, memory_order order);
}
\end{codeblock}

%%\uline{\texttt{\ namespace\ std::experimental::inline\ %%concurrency\_v2\ \{\ \ ~~void*\ %%atomic\_load\_per\_byte\_memcpy(void*\ dest,\ const\ void*\ %%source,\ size\_t\ count,\ memory\_order\ order);\ \ ~~void*\ %%atomic\_store\_per\_byte\_memcpy(void*\ dest,\ const\ void*\ %%source,\ size\_t\ count,\ memory\_order\ order);\ \ ~~\#define\ %%\_\_cpp\_lib\_experimental\_bytewise\_atomic\_memcpy\ \ %%202XYYL\ \}}}
%%\rSec1[bytewiseatomicmemcpy.general]{General}
\pnum

The \tcode{atomic_load_per_byte_memcpy()} and
\tcode{atomic_store_per_byte_memcpy()} functions support concurrent
programming idioms in which values may be read while being written, but
the value is trusted only when it can be determined after the fact that
a race did not occur. \begin{note} So-called "seqlocks" are the canonical
example of such an idiom. \end{note}


\rSec1[atomiccpy.load]{\tcode{atomic_load_per_byte_memcpy}}
\pnum
The \tcode{atomic_load_per_byte_memcpy} /
\tcode{atomic_store_per_byte_memcpy} functions behave as if the
\tcode{source} and \tcode{dest} bytes respectively were individual
atomic objects.

\textbf{\tcode{void*\ atomic_load_per_byte_memcpy(void* dest, const void* source, size_t count, memory_order order);}}

\begin{itemdescr}
\pnum
\expects

\tcode{order} is \tcode{memory_order::acquire} or
\tcode{memory_order::relaxed}. \tcode{(char*)dest + [0, count)}
and \tcode{(const char*)source + [0, count)} are valid ranges
that do not overlap.

\pnum
\effects
Copies \tcode{count} consecutive bytes pointed to by
\tcode{source} into consecutive bytes pointed to by \tcode{dest}. Each
individual load operation from a source byte is atomic with memory order
\tcode{order}. These individual loads are unsequenced with respect to
each other. The function implicitly creates objects (\CppXref{6.7.2})
in the destination region of storage immediately prior to copying the
sequence of bytes to the destination. 
\begin{note} There is no requirement
that the individual bytes be copied in order, or that the implementation
must operate on individual bytes. \end{note}

\pnum
\returns
\tcode{dest}.
\end{itemdescr}

\rSec1[atomiccpy.store]{\tcode{atomic_store_per_byte_memcpy}}
\textbf{\tcode{void* atomic_store_per_byte_memcpy(void* dest, const void* source, size_t count, memory_order order);}}

\begin{itemdescr}
\pnum
\expects
\tcode{order} is \tcode{memory_order::release} or
\tcode{memory_order::relaxed}. \tcode{(char*)dest + [0, count)}
and \tcode{(const char*)source + [0, count)} are valid ranges
that do not overlap.

\pnum
\effects
Copies \tcode{count} consecutive bytes pointed to by
\tcode{source} into consecutive bytes pointed to by \tcode{dest}. Each
individual store operation to a destination byte is atomic with memory
order \tcode{order}. These individual stores are unsequenced with
respect to each other. The function implicitly creates objects
(\CppXref{6.7.2}) in the destination region of storage immediately
prior to copying the sequence of bytes to the destination.

\pnum
\returns

\tcode{dest}.
\end{itemdescr}

\begin{note} If any of the atomic byte loads performed by an
\tcode{atomic_load_per_byte_memcpy()} call A with
\tcode{memory_order::acquire} argument take their value from an atomic
byte store performed by \tcode{atomic_store_per_byte_memcpy()} call
B with \tcode{memory_order::release} argument, then the start of B
strongly happens before the completion of A. 
\end{note}  
%%\end{quote}
