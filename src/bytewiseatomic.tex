%!TEX root = ts.tex

\rSec0[bytewiseatomicmemcpy]{Bytewise Atomic Memcpy}

\rSec1[bytewiseatomicmemcpy.general]{General}

This clause describes bytewise atomic memcpy access. 


\rSec1[bytewiseatomicmemcpy.syn]{Header <bytewiseatomicmemcpy> synopsis}
%%\uline{\textbf{Header
%%\texttt{\textless{}experimental/bytewise\_atomic\_memcpy\textgre%%ater{}}
%%synopsis}}

\begin{codeblock}
namespace std::experimental::inline concurrency_v2 {

  void* atomic_load_per_byte_memcpy(void* dest, const void* source, size_t count, memory_order order);

  void* atomic_store_per_byte_memcpy(void* dest, const void* source, size_t count, memory_order order);


}
\end{codeblock}

%%\uline{\texttt{\ namespace\ std::experimental::inline\ %%concurrency\_v2\ \{\ \ ~~void*\ %%atomic\_load\_per\_byte\_memcpy(void*\ dest,\ const\ void*\ %%source,\ size\_t\ count,\ memory\_order\ order);\ \ ~~void*\ %%atomic\_store\_per\_byte\_memcpy(void*\ dest,\ const\ void*\ %%source,\ size\_t\ count,\ memory\_order\ order);\ \ ~~\#define\ %%\_\_cpp\_lib\_experimental\_bytewise\_atomic\_memcpy\ \ %%202XYYL\ \}}}
%%\rSec1[bytewiseatomicmemcpy.general]{General}
\pnum

The \texttt{atomic\_load\_per\_byte\_memcpy()} and
\texttt{atomic\_store\_per\_byte\_memcpy()} functions support concurrent
programming idioms in which values may be read while being written, but
the value is trusted only when it can be determined after the fact that
a race did not occur. \begin{note} So-called \"seqlocks\" are the canonical
example of such an idiom. \end{note}


\rSec1[bytewiseatomicmemcpy.load]{atomic_load_per_byte_memcpy}
\pnum
The \texttt{atomic\_load\_per\_byte\_memcpy} /
\texttt{atomic\_store\_per\_byte\_memcpy} functions behave as if the
\texttt{source} and \texttt{dest} bytes respectively were individual
atomic objects.

\textbf{\texttt{void*\ atomic\_load\_per\_byte\_memcpy(void*\ dest,\ const\ void*\ source,\ size\_t\ count,\ memory\_order\ order);}}

\begin{itemdescr}
\pnum
\expects

\texttt{order} is \texttt{memory\_order::acquire} or
\texttt{memory\_order::relaxed}. \texttt{(char*)dest\ +\ {[}0,\ count)}
and \texttt{(const\ char*)source\ +\ {[}0,\ count)} are valid ranges
that do not overlap.

\pnum
\effects
Copies \texttt{count} consecutive bytes pointed to by
\texttt{source} into consecutive bytes pointed to by \texttt{dest}. Each
individual load operation from a source byte is atomic with memory order
\texttt{order}. These individual loads are unsequenced with respect to
each other. The function implicitly creates objects ({[}intro.object{]})
in the destination region of storage immediately prior to copying the
sequence of bytes to the destination. 
\begin{note} There is no requirement
that the individual bytes be copied in order, or that the implementation
must operate on individual bytes. \end{note}

\pnum
\returns
\texttt{dest}.
\end{itemdescr}

\rSec1[bytewiseatomicmemcpy.store]{atomic_store_per_byte_memcpy}
\textbf{\texttt{void*\ atomic\_store\_per\_byte\_memcpy(void*\ dest,\ const\ void*\ source,\ size\_t\ count,\ memory\_order\ order);}}

\begin{itemdescr}
\pnum
\expects
\texttt{order} is \texttt{memory\_order::release} or
\texttt{memory\_order::relaxed}. \texttt{(char*)dest\ +\ {[}0,\ count)}
and \texttt{(const\ char*)source\ +\ {[}0,\ count)} are valid ranges
that do not overlap.

\pnum
\effects
Copies \texttt{count} consecutive bytes pointed to by
\texttt{source} into consecutive bytes pointed to by \texttt{dest}. Each
individual store operation to a destination byte is atomic with memory
order \texttt{order}. These individual stores are unsequenced with
respect to each other. The function implicitly creates objects
({[}intro.object{]}) in the destination region of storage immediately
prior to copying the sequence of bytes to the destination.

\pnum
\returns

\texttt{dest}.
\end{itemdescr}

\begin{note} If any of the atomic byte loads performed by an
\texttt{atomic\_load\_per\_byte\_memcpy()} call A with
\texttt{memory\_order::acquire} argument take their value from an atomic
byte store performed by \texttt{atomic\_store\_per\_byte\_memcpy()} call
B with \texttt{memory\_order::release} argument, then the start of B
strongly happens before the completion of A. 
\end{note}  
%%\end{quote}