%!TEX root = ts.tex

\rSec0[foreword]{Foreword}
%%\hypersetup{colorlinks=true, linkcolor=blue, urlcolor=blue}

ISO (the International Organization for Standardization) and IEC (the International Electrotechnical Commission) form the specialized system for worldwide standardization. National bodies that are members of ISO or IEC participate in the development of International Standards through technical committees established by the respective organization to deal with particular fields of technical activity. ISO and IEC technical committees collaborate in fields of mutual interest. Other international organizations, governmental and non-governmental, in liaison with ISO and IEC, also take part in the work.

The procedures used to develop this document and those intended for its further maintenance are described in the ISO/IEC Directives, Part 1. In particular, the different approval criteria needed for the different types of document should be noted. This document was drafted in accordance with the editorial rules of the ISO/IEC Directives, Part 2 (see \href{www.iso.org/directives}{www.iso.org/directives} 
or 
\href{www.iec.ch/members_experts/refdocs}{www.iec.ch/members_experts/refdocs}).

ISO and IEC draw attention to the possibility that the implementation of this document may involve the use of (a) patent(s). ISO and IEC take no position concerning the evidence, validity or applicability of any claimed patent rights in respect thereof. As of the date of publication of this document, ISO and IEC had not received notice of (a) patent(s) which may be required to implement this document. However, implementers are cautioned that this may not represent the latest information, which may be obtained from the patent database available at \href{www.iso.org/patents}{www.iso.org/patents} 
and 
\href{patents.iec.ch/}{patents.iec.ch}. 
ISO and IEC shall not be held responsible for identifying any or all such patent rights.

Any trade name used in this document is information given for the convenience of users and does not constitute an endorsement.

For an explanation of the voluntary nature of standards, the meaning of ISO specific terms and expressions related to conformity assessment, as well as information about ISO's adherence to the World Trade Organization (WTO) principles in the Technical Barriers to Trade (TBT) see \href{www.iso.org/iso/foreword.html/}{www.iso.org/iso/foreword.html}. In the IEC, see \href{www.iec.ch/understanding-standards/}{www.iec.ch/understanding-standards}.

This document was prepared by Joint Technical Committee ISO/IEC JTC 1, \emph{Information technology}, Subcommittee SC 22, \emph{Programming languages, their environments and system software interfaces}.

Any feedback or questions on this document should be directed to the user’s national standards body. A complete listing of these bodies can be found at \href{www.iso.org/members.html/}{www.iso.org/members.html} and \href{www.iec.ch/national-committees/}{www.iec.ch/national-committees}.
