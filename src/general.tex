
\rSec0[intro]{General}

@@@ Michael Wong to fill in and update.

\rSec1[general.scope]{Scope}
\pnum
This Technical Specification describes requirements for implementations of an interface that computer programs written in the C++ programming language may use to invoke algorithms with concurrent execution. The algorithms described by this Technical Specification are realizable across a broad class of computer architectures.

\pnum
This Technical Specification is non-normative. Some of the functionality described by this Technical Specification may be considered for standardization in a future version of C++, but it is not currently part of any C++ standard. Some of the functionality in this Technical Specification may never be standardized, and other functionality may be standardized in a substantially changed form.

\pnum
The goal of this Technical Specification is to build widespread existing practice for concurrency in the C++ standard algorithms library. It gives advice on extensions to those vendors who wish to provide them.

\rSec1[general.refs]{Normative References}


\rSec1[general.def]{Terms and definitions}

\rSec1[general.compliance]{Implementation compliance}

\rSec1[general.namespaces]{Namespaces and headers and modifications to standard classes}

\rSec1[general.feature.test]{Feature-testing recommendations (Informative)}

\rSec1[general.plans]{Future plans (Informative)}

\rSec1[general.ack]{Acknowledgments}

This work is the result of a collaboration of researchers in industry and academia. We wish to thank the
original authors of this TS, Michael Wong, Paul McKenney, and Maged Michael. We also wish to thank people
who made valuable contributions within and outside these groups, including Jens Maurer, and many others not named
here who contributed to the discussion.




