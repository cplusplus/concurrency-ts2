%!TEX root = ts.tex

\rSec1[saferecl.rcu]{Read-copy update (RCU)}

\rSec2[saferecl.rcu.general]{RCU general}

\pnum
RCU is a synchronization mechanism that can be used for linked data
structures that are frequently read, but seldom updated.  RCU does
not provide mutual exclusion, but instead allows the user to schedule
specified actions such as deletion at some later time.

\pnum
A class type \tcode{T} is \defn{rcu-protectable} if it has exactly one
public base class of type \tcode{rcu_obj_base<T,D>} for some \tcode{D}
and no base classes of type \tcode{rcu_obj_base<X,Y>} for any other
combination \tcode{X}, \tcode{Y}. An object is rcu-protectable if it is
of rcu-protectable type.

\pnum
An invocation of \tcode{unlock U} on an \tcode{rcu_domain dom} corresponds
to an invocation of \tcode{lock L} on \tcode{dom} if \tcode{L} is
sequenced before \tcode{U} and either

\begin{itemize}
\item	no other invocation of \tcode{lock} on \tcode{dom} is sequenced
	after \tcode{L} and before \tcode{U} or
\item	every invocation of \tcode{unlock U'} on \tcode{dom} such
	that \tcode{L} is sequenced before \tcode{U'} and \tcode{U'}
	is sequenced before \tcode{U} corresponds to an invocation of
	\tcode{lock L'} on \tcode{dom} such that \tcode{L} is sequenced
	before \tcode{L'} and \tcode{L'} is sequenced before \tcode{U'}.
\end{itemize}

% @@@ In the draft standard, the "note" environment generates the
% @@@ "[ Note: blahblahblah. --- end note. ]" sequence.  It is a bit
% @@@ different here.  I am using it anyway.

\pnum
\begin{note}
This pairs nested locks and unlocks on a given domain in each thread.
\end{note}

\pnum
A \defn{region of RCU protection} on a domain \tcode{dom} starts
with a \tcode{lock L} on \tcode{dom} and ends with its corresponding
\tcode{unlock U}.

\pnum
Given a region of RCU protection \tcode{R} on a domain \tcode{dom}
and given an evaluation \tcode{E} that scheduled another evaluation
\tcode{F} in \tcode{dom}, if \tcode{E} does not strongly happen before
the start of \tcode{R}, the end of \tcode{R} strongly happens before
evaluating \tcode{F}.

\pnum
The evaluation of a scheduled evaluation is potentially concurrent with
any other such evaluation. Each scheduled evaluation is evaluated at
most once.

\rSec2[saferecl.rcu.syn]{RCU header \tcode{<rcu>} synopsis}

Editor's note: This section was mistakenly unnumbered in P1122R4.

% @@@ Copy and paste from Google Documents gets space characters and
% @@@ "<>" characters that LaTeX really hates.  Just replace them.

% \indexheader{rcu}%
% @@@ Added missing "," after a number of the references in this code block.
\begin{codeblock}
namespace std::experimental::inline concurrency_v2 {
  // \ref{saferecl.rcu.base}, class template rcu_obj_base
  template<class T, class D = default_delete<T>>
    class rcu_obj_base;

  // \ref{saferecl.rcu.domain}, class rcu_domain
  class rcu_domain;

  // \ref{saferecl.rcu.default.domain}, rcu_default_domain
  rcu_domain& rcu_default_domain() noexcept;

  // \ref{saferecl.rcu.synchronize}, rcu_synchronize
  void rcu_synchronize(rcu_domain& dom = rcu_default_domain()) noexcept;

  // \ref{saferecl.rcu.barrier}, rcu_barrier
  void rcu_barrier(rcu_domain& dom = rcu_default_domain()) noexcept;

  // \ref{saferecl.rcu.retire}, rcu_retire
  template<class T, class D = default_delete<T>>
    void rcu_retire(T* p, D d = D(), rcu_domain& dom = rcu_default_domain());
}
\end{codeblock}

\rSec2[saferecl.rcu.base]{Class \tcode{rcu_obj_base}}

% @@@ Added missing "of" in following sentence.

Objects of type \tcode{T} to be protected by RCU inherit from a
specialization of \tcode{rcu_obj_base<T,D>}.

% @@@ Removed obsolete in-code reference to the now-nonexistent
% @@@ separate section for rcu_obj_base::retire.

\begin{codeblock}
template<class T, class D = default_delete<T>>
class rcu_obj_base {
public:
  void retire(D d = D(), rcu_domain& dom = rcu_default_domain()) noexcept;
protected:
  rcu_obj_base() = default;
private:
  D deleter;  // exposition only
};
\end{codeblock}

\pnum
A client-supplied template argument \tcode{D} shall be a
function object type \tcode{([function.object])} for which,
given a value \tcode{d} of type \tcode{D} and a value \tcode{ptr}
of type \tcode{T*}, the expression \tcode{d(ptr)} is valid and
has the effect of disposing of the pointer as appropriate for
that deleter.

\pnum
The behavior of a program that adds specializations for
\tcode{rcu_obj_base} is undefined.

\pnum
\tcode{D} shall meet the requirements for
Cpp17DefaultConstructible and Cpp17MoveAssignable.

\pnum
\tcode{T} may be an incomplete type.

\pnum
If \tcode{D} is trivially copyable, all specializations of
\tcode{rcu_obj_base<T,D>} are trivially copyable.

\begin{itemdecl}
void retire(D d = D(), rcu_domain& dom = rcu_default_domain()) noexcept;
\end{itemdecl}

\begin{itemdescr}

\pnum
\defn{Mandates}: \tcode{T} is an rcu-protectable type.

\pnum
\defn{Preconditions}: \tcode{*this} is a base class subobject of
an object \tcode{x} of type \tcode{T}. The member function
\tcode{rcu_obj_base<T,D>::retire} was not invoked on \tcode{x}
before. The assignment to \tcode{deleter} does not throw an
exception. The expression \tcode{deleter(addressof(x))} has
well-defined behavior and does not throw an exception.

\pnum	\defn{Effects}: Evaluates \tcode{deleter = std::move(d)} and schedules
the evaluation of the expression \tcode{de\-let\-er(ad\-dress\-of(x))} in the domain \tcode{dom}.

\pnum	\defn{Remarks}: It is implementation-defined whether or not scheduled
evaluations in \tcode{dom} can be invoked by the \tcode{retire}
function.
\begin{note}
If such evaluations acquire resources held across any invocation of
retire on \tcode{dom}, deadlock can occur.
\end{note}

\end{itemdescr}

\rSec2[saferecl.rcu.domain]{Class \tcode{rcu_domain}}

% @@@ Removed the obsolete reference comments.

This class meets the requirements of Cpp17BasicLockable \tcode{[thread.req.lockable.basic]} and provides regions of RCU protection, for example, as follows:

\begin{codeblock}
std::scoped_lock<rcu_domain> rlock(rcu_default_domain());
\end{codeblock}

\begin{codeblock}
class rcu_domain {
public:
  rcu_domain(const rcu_domain&) = delete;
  rcu_domain& operator=(const rcu_domain&) = delete;

  void lock() noexcept;
  void unlock() noexcept;
};
\end{codeblock}

The functions \tcode{lock} and \tcode{unlock} establish (possibly nested)
regions of RCU protection.

\rSec3[saferecl.rcu.domain.lock]{\tcode{rcu_domain::lock}}

\begin{itemdecl}
void lock() noexcept;
\end{itemdecl}

\begin{itemdescr}

\pnum
\defn{Effects}: Opens a region of RCU protection.

\pnum
\defn{Remarks}: Calls to the function lock do not introduce a data race
\tcode{[intro.races]} involving \tcode{*this}.

\end{itemdescr}

\rSec3[saferecl.rcu.domain.unlock]{\tcode{rcu_domain::unlock}}

\begin{itemdecl}
void unlock() noexcept;
\end{itemdecl}

\begin{itemdescr}

\pnum
\defn{Preconditions}: A call to the function \tcode{lock} that opened
an unclosed region of RCU protection is sequenced before the
call to \tcode{unlock}.

\pnum
\defn{Effects}: Closes the unclosed region of RCU protection that was
most recently opened.

\pnum
\defn{Remarks}:   It is implementation-defined whether or not scheduled
evaluations in \tcode{*this} can be invoked by the \tcode{unlock}
function.
\begin{note}
If such evaluations acquire resources held across any invocation
of \tcode{unlock} on \tcode{*this}, deadlock can occur.
\end{note}
Calls to the function \tcode{unlock} do not introduce a data race
(16.4.6.10) involving \tcode{*this}.
\begin{note}
Evaluation of scheduled evaluations can still cause a data race.
\end{note}

\end{itemdescr}

\rSec2[saferecl.rcu.default.domain]{\tcode{rcu_default_domain}}

\begin{itemdecl}
rcu_domain& rcu_default_domain() noexcept;
\end{itemdecl}

\begin{itemdescr}

\pnum
\defn{Returns}: A reference to the default object of type \tcode{rcu_domain}.
A reference to the same object is returned every time this
function is called.

\end{itemdescr}

\rSec2[saferecl.rcu.synchronize]{\tcode{rcu_synchronize}}

\begin{itemdecl}
void rcu_synchronize(rcu_domain& dom = rcu_default_domain()) noexcept;
\end{itemdecl}

\begin{itemdescr}

\pnum
\defn{Effects}: If the call to \tcode{rcu_synchronize} does not strongly
happen before the lock opening an RCU protection region \tcode{R}
on \tcode{dom}, blocks until the \tcode{unlock} closing \tcode{R}
happens.

\pnum
\defn{Synchronization}: The \tcode{unlock} closing \tcode{R} strongly
happens before the return from \tcode{rcu_synchronize}.

\end{itemdescr}

\rSec2[saferecl.rcu.barrier]{\tcode{rcu_barrier}}

\begin{itemdecl}
void rcu_barrier(rcu_domain& dom = rcu_default_domain()) noexcept;
\end{itemdecl}

\begin{itemdescr}

\pnum
\defn{Effects}: May evaluate any scheduled evaluations in
\tcode{dom}.  For any evaluation that happens before the call
to \tcode{rcu_barrier} and that schedules an evaluation \tcode{E}
in \tcode{dom}, blocks until \tcode{E} has been evaluated.  

\pnum
\defn{Synchronization}: The evaluation of any such \tcode{E} strongly
happens before the return from \tcode{rcu_barrier}.

\end{itemdescr}

\rSec2[saferecl.rcu.retire]{Template \tcode{rcu_retire}}

\begin{itemdecl}
template<class T, class D = default_delete<T>>
void rcu_retire(T* p, D d = D(), rcu_domain& dom = rcu_default_domain());
\end{itemdecl}

\begin{itemdescr}

\pnum
\defn{Mandates}: \tcode{is_move_constructible_v<D>} is true.

\pnum
\defn{Preconditions}: \tcode{D} meets the Cpp17MoveConstructible and
Cpp17Destructible requirements.
The expression \tcode{d1(p)}, where \tcode{d1} is defined below, is
well-formed and its evaluation does not exit via an exception.

\pnum
\defn{Effects}: May allocate memory.
It is unspecified whether the memory allocation is performed by
invoking \tcode{operator} \tcode{new}.
Initializes an object \tcode{d1} of type \tcode{D} from
\tcode{std::move(d)}.
Schedules the evaluation of \tcode{d1(p)} in the domain
\tcode{dom}.
\begin{note}
If \tcode{rcu_retire} exits via an exception, no evaluation
is scheduled.
\end{note}

\pnum
\defn{Throws}: Any exception that would be caught by a handler of type
\tcode{bad_alloc}.
Any exception thrown by the initialization of \tcode{d1}.

\pnum
\defn{Remarks}: It is implementation-defined whether or not scheduled
evaluations in dom can be invoked by the \tcode{rcu_retire}
function.
\begin{note}
If such evaluations acquire resources held across any invocation
of \tcode{rcu_retire} on \tcode{dom}, deadlock can occur.
\end{note}

\end{itemdescr}
